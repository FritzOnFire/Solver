\documentclass{article}

\title{(short) Report on N puzzle solver.}
\author{F Engelbrecht}
\date{13 May 2015}

\begin{document}
	\maketitle

	\section{Introduction.}

	The problem is to solve a N puzzle problem in the least amount of moves and
	to find out what those moves are in the least amount of time.

	\section{Design and Implementation.}

	\subsection{Algorithm for checking unsolvable.}

	Create two Linklists. The first Linklist has two values, weight and value.
	The first node has a value of 1 and the last node has a value of N*N-1 (N
	is the board width/height). The weight for each node is 0. Read through 
	the current board.  As a number is read travers through the linklist 
	incrementing each node weight until a node with the same value as the 
	number read. Pop off the value and add it to the second Linklist.
	\\
	If N*N is odd and the sum of all the weights in the second Linklist is even
	then the board is solvable. If N*N is even and the sum of the sum of the
	weights in the second Linklist and the row position is odd then the board
	is also solvable. If the board does not fulfill these requirements then the
	board is unsolvable.

	\subsection{Algorithm for solving.}

	A Linklist is created with the following values. A two dimensional array 
	that stores the current board state. A value that stores the current weight
	of the board. A value that stores the amount of moves made by this current
	state. Two values that store the y and x coordinates of the empty space. A
	Linklist of previous moves and a pointer to another state in the list.
	\\
	The node at the head of the Linklist is popped off and then all possible 
	moves are played on the board and saved as new states. The states are then
	added to the Linklist in ascending order determined by the weight. The top 3
	nodes are then evaluated to see if there are finished boards among them. The
	process is then repeated until a finished board is found.

	\subsection{}

\end{document}
